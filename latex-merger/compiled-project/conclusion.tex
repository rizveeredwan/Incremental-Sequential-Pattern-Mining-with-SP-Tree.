\section{Conclusions} \label{conclusion}
In this study, we have proposed two novel tree-based solutions, Tree-Miner based on SP-Tree and IncTree-Miner based on IncSP-Tree to solve the SPM problem for static and incremental databases respectively. The tree-based structure provides structural advantage which ultimately helps to improve mining performance and handle manipulation over the database. We have also presented a new breadth-first based support counting technique which helps detect the infrequent patterns early and a heuristic pruning strategy to reduce redundant search space. We have also discussed the newly proposed pattern storage structure BPFSP-Tree for the ISPM problem based on IncSP-Tree and its efficient bottom-up pruning strategy. Our proposed solutions are designed based on a single support threshold parameter and able to mine the complete set of frequent sequential patterns having ``build-once-mine-many" property leading to also being suitable for interactive mining. 


Moreover, we have discussed the extendability of our approach to other solutions, such as, memory resilient version. We have explored various aspects related to the proposed solutions' implementations which ultimately lights up on the solutions' flexibilities. We have also discussed different challenges related to the incremental mining problem, e.g., usage of sequence summarizer to incrementally update the co-occurrence information of the complete database, concept drift issues, etc. In the performance evaluation section, we have provided analysis for both of our solutions and showed their efficiency in improving mining runtime. As an ongoing and future work, we have planned to extend our solution to solve the SPM problem for data streams, parallel and distributed environments and specialized attribute based databases, such as, weighted and uncertain databases. We also have planned to modify our tree-based solutions to approach the specialized sequential pattern discovery problems, e.g, maximal patterns, closed patterns, top-k patterns, etc. using our novel tree structures' properties and utilities.  
